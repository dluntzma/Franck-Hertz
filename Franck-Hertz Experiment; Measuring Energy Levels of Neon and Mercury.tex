\documentclass[prb,preprint]{revtex4-1} 

\usepackage{amsmath}  
\usepackage{amsfonts} 
\usepackage{graphicx} 

\begin{document}


\title{Franck-Hertz Experiment; Measuring Energy Levels of Neon and Mercury}

\author{Jiajun}
\email{XXX} 
\affiliation{XXX}


\author{Danika Luntz-Martin}
\email{dluntzma@smith.edu}
\affiliation{Department of Physics, Smith College, Northampton, MA , 01063}


\date{\today}

\begin{abstract}


\end{abstract}


\maketitle 


\section{Introduction} 


\section{Methods}

\subsection{Neon}
\subsection{Mercury}


\section{Results}

\subsection{Neon}

We did four data runs using neon. For each run we recorded the accelerating voltage (x data) and the electron current measured by the anode???, this current was recorded as a voltage measured across an internal resister. Each of our runs showed three discernible dips in the voltage corresponding to electron current, see the sample data run Figure ~\ref{neon_data}. These dips are the voltages just before the electrons have enough energy (from the accelerating voltage) to reach the anode even after an inelastic collision with a neon atom. The apparent double minima, see the second dip in Figure ~\ref{neon_data}, is most likely caused by by energy levels with very similar excitation energies.~\cite{XXX} Also of interest is the voltage corresponding to the steepest negative slope which is when the majority of electrons have enough energy to cite the neon atoms. However the location of the steepest negative slope was difficult to determine from our data, again see Figure ~\ref{neon_data}.

\subsection{Mercury}

\section{Analysis}

\subsection{Neon}
\subsection{Mercury}

\section{Discussion}


\section{Conclusion}



\begin{thebibliography}{9}


\bibitem{latexsite} \LaTeX\ Project Web Site, \url{<http://www.latex-project.org/>}.

\bibitem{latexbook}Helmut Kopka and Patrick W. Daly, \textit{A Guide to
\LaTeX}, 4th edition (Addison-Wesley, Boston, 2004).

\bibitem{revtex} REV\TeX\ 4 Home Page, \url{<https://authors.aps.org/revtex4/>}.

\bibitem{cloudLaTeX} On the other hand, you can avoid the installation process
entirely by using a cloud-based \LaTeX\ processor such as ShareLaTeX,
\url{<https://www.sharelatex.com/>}, or write\LaTeX, \url{<https://www.writelatex.com/>}.

\bibitem{mermin} N. David Mermin, ``What's wrong with these equations?,'' 
Phys. Today \textbf{42} (10), 9--11 (1989).  
% Note that the issue number (10) in this citation is required, because
% each issue of Physics Today starts over with page 1.  Also note the use of
% an en-dash (--), not a hyphen (-), for the page range.

\bibitem{feynman} Richard P. Feynman, Robert B. Leighton, and Matthew Sands, 
\textit{The Feynman Lectures on Physics, Vol.\ 1} (Addison-Wesley, 1964), p.~3-10.
% Note that this book is paginated by chapter; "3-10" is a single page reference
% that uses a hyphen, not a range of pages that would us an en-dash (--).


\end{thebibliography}


\end{document}
